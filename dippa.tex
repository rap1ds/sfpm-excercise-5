\documentclass[a4paper]{article}

% Encoodaus, joka sopii suomenkielellä (esim. ä ja ö)
\usepackage[utf8]{inputenc}
\usepackage[T1]{fontenc}

% Suomenkielinen tavutus
\usepackage[finnish]{babel}

% Viitteet
\usepackage{natbib}

% Otsikkojen päätteetön fontti
\usepackage{sectsty}
\allsectionsfont{\sffamily\large}

% Viitteiden merkit
\bibpunct{(}{)}{;}{a}{,}{,}

\begin{document}

\title{\small T-76.5612 Software Project Management \\ Exercise 5, Pre-Lecture \\ \huge Ihmisten johtaminen}
\date{20.2.2012}
\author{Mikko Koski \\ mikko.koski@aalto.fi \\ 66467F}
\maketitle

\normalsize

\section{Johdanto}

Ohjelmistokehitysprojektiin osallistuu monia henkilöitä, joilla kaikilla on oma roolinsa projektissa. Projektin ytimen luo kehitystiimi, mutta tiimin ympärillä on joukko muita henkilöitä omine rooleineen. Tällaisia projektitiimin ulkopuolella olevia rooleja ovat esimerkiksi projektimanageri, Scrum master, Agile-valmentaja sekä useimmiten asiakkasorganisaatiota edustava tuotteen omistaja (product owner).

Ketterät ohjelmistokehitysmenetelmät ovat muokanneet projektityöskentelyä myös roolitusten osalta. Perinteisesti projektit ovat koostuneet valtahierarkiassa ylempänä olevasta projektimanagerista sekä alempana olevasta tiimistä. Ketterien menetelmien mukanaan tuovat uudet roolit, muun muassa Scrum master ja Agile-valmentaja, eivät sen sijaan ole hierarkiassa ylempänä vaan pikemminkin samalla tasolla. Scrum master ja Agile-valmentaja eivät esimerkiksi käytä käskytysvaltaa tiimiin, vaan pyrkivät sen sijaan tehostamaan tiimin työskentelyä yhteistyön kautta.

Tämän esseen tarkoituksena on vertailla eri roolien merkitystä projekteissa sekä keskustella yleisesti ihmisten johtamisesta IT-alalla.

\section{Projektitiimin jäsenten eri roolit}

\subsection{Projektimanageri}

Projektimanageri tittelinä kuulostaa nykypäivänä ketterien menetelmien vallitessa hieman vanhahtavalta. Tämä johtunee siitä, että esimerkiksi Scrum-projektiin kuuluu tuotteen omistaja, Scrum master sekä tiimi, muttei ollenkaan niin sanottua projektimanageria. Ketterät menetelmät eivät selkeästi määrittele, mikä on projektimanagerin rooli \citep{augustine2005}.

Perinteisesti projektimanageri on johtanut suunnitelmavetoisesti. Tämä tarkoittaa sitä, että projektin alussa luodaan suunnitelma projektin kulusta ja tätä suunnitelmaa pyritään mahdollisimman tarkasti noudattamaan. Projektimanagerin vastuulla on suunnitelman noudattaminen. Projekti on menestynyt, jos sovittu tuote toimitetaan ajallaan, budjettirajojen sisäpuolella sovitussa laajuudessaan \citep{adkins2010}.

Projektimanagerilla on myös vastuu projektin liiketoiminnallisesta kannattavuudesta. Tämän lisäksi projektimanageri on vastuussa projektin kokonaisonnistumisesta tai epäonnistumisesta \citep{augustine2005}.



\subsection{Scrum Master}

\subsection{Agile-valmentaja}

\subsection{Tuotteen omistaja}



% Discuss the differences between the following roles in software development projects: 
% - project manager
% - product owner
% - scrum master
% - agile coach

% What are the most important tasks of each of these roles? 

% How are these roles different? 

% Discuss especially from the point of view on how these different roles are involved in leading and managing people.





% Business vastuu?!?!?!

\section{Ihmisten johtaminen IT-alalla}

% Discuss what can be difficult when managing/leading software developers and building software development teams. 

% Feel free to add questions to our visiting lecturers regarding this topic in the end of your answer.

\citep{grosjean2010}
\citep{appelo2012}
\citep{rsaanimate}
\citep{adkins2010}
\citep{augustine2005}
\citep{mcconnell1996}

\bibliographystyle{plainnat}
\bibliography{ref}

\end{document}