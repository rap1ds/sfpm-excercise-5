\documentclass[a4paper]{article}

% Encoodaus, joka sopii suomenkielellä (esim. ä ja ö)
\usepackage[utf8]{inputenc}
\usepackage[T1]{fontenc}

% Suomenkielinen tavutus
\usepackage[finnish]{babel}

% Viitteet
\usepackage{natbib}

% Otsikkojen päätteetön fontti
\usepackage{sectsty}
\allsectionsfont{\sffamily\large}

% Viitteiden merkit
\bibpunct{(}{)}{;}{a}{,}{,}

\begin{document}

\title{\small T-76.5612 Software Project Management \\ Exercise 5, Pre-Lecture \\ \huge Ihmisten johtaminen}
\date{20.2.2012}
\author{Mikko Koski \\ mikko.koski@aalto.fi \\ 66467F}
\maketitle

\normalsize

\section{Johdanto}

Ohjelmistokehitysprojektiin osallistuu monia henkilöitä, joilla kaikilla on oma roolinsa projektissa. Projektin ytimen luo kehitystiimi, mutta tiimin ympärillä on joukko muita henkilöitä omine rooleineen. Tällaisia projektitiimin ulkopuolella olevia rooleja ovat esimerkiksi projektimanageri, Scrum master, Agile-valmentaja sekä useimmiten asiakkasorganisaatiota edustava tuotteen omistaja (product owner).

Ketterät ohjelmistokehitysmenetelmät ovat muokanneet projektityöskentelyä myös roolitusten osalta. Perinteisesti projektit ovat koostuneet valtahierarkiassa ylempänä olevasta projektimanagerista sekä alempana olevasta tiimistä. Ketterien menetelmien mukanaan tuovat uudet roolit, muun muassa Scrum master ja Agile-valmentaja, eivät sen sijaan ole hierarkiassa ylempänä vaan pikemminkin samalla tasolla. Scrum master ja Agile-valmentaja eivät esimerkiksi käytä käskytysvaltaa tiimiin, vaan pyrkivät sen sijaan tehostamaan tiimin työskentelyä yhteistyön kautta.

Tämän esseen tarkoituksena on vertailla eri roolien merkitystä projekteissa sekä keskustella yleisesti ihmisten johtamisesta IT-alalla.

\section{Projektitiimin jäsenten eri roolit}

\subsection{Projektimanageri}

Projektimanageri tittelinä kuulostaa nykypäivänä ketterien menetelmien vallitessa hieman vanhahtavalta. Tämä johtunee siitä, että esimerkiksi Scrum-projektiin kuuluu tuotteen omistaja, Scrum master sekä tiimi, muttei ollenkaan niin sanottua projektimanageria. Ketterät menetelmät eivät selkeästi määrittele, mikä on projektimanagerin rooli \citep{augustine2005}.

Perinteisesti projektimanageri on johtanut suunnitelmavetoisesti. Tämä tarkoittaa sitä, että projektin alussa luodaan suunnitelma projektin kulusta ja tätä suunnitelmaa pyritään mahdollisimman tarkasti noudattamaan. Projektimanagerin vastuulla on suunnitelman noudattaminen. Projekti on menestynyt, jos sovittu tuote toimitetaan ajallaan, budjettirajojen sisäpuolella sovitussa laajuudessaan \citep{adkins2010}.

Projektimanagerilla on myös vastuu projektin liiketoiminnallisesta kannattavuudesta. Tämän lisäksi projektimanageri on vastuussa projektin kokonaisonnistumisesta tai epäonnistumisesta \citep{augustine2005}.

Projektimanagerin vastuut onnistumisesta vaikuttavat hänen asemaansa muuhun tiimiin nähden. Vastuusta johtuen projektimanageri onkin hierarkiassa tiimin jäseniä ylempänä ja hänellä on myös oikeus käyttää käskytysvaltaa tiimiä kohtaan. Tämä vaikuttaa tapaan, jolla projektimanageri tiimin jäseniä johtaa.

Koska projektimanageri ottaa vastuun projektin onnistumisesta ei tiimi joudu tätä vastuuta ottamaan. Tämä saattaa olla tiimin kannalta hieman epämotivoivaa. Mielestäni olisi luontevampaa, että tiimi, joka loppukädessä on vastuussa toteutuksesta olisi myös vastuussa projektin kokonaisonnistumisesta.

\subsection{Scrum Master}

Scrum Masterin rooli eroaa hyvin paljon projektimanagerin roolista. Scrum master ei ole esimerkiksi vastuussa projektin onnistumisesta, toisin kuin projektimanageri. Ketterien menetelmien mukaan tiimi on vastuussa projektin menestymisestä.

Koska Scrum Master ei ole vastuussa projektin onnistumisesta, asettaa se hänet myös hyvin erilaiseen asemaan tiimiin nähden verrattuna projektimanageriin. Scrum Master ei ole hierarkiassa korkeammalla tiimiin nähden vaan pikemminkin samalla tasolla. Scrum Masterilla ei myöskään ole käskytysvaltaa tiimiä kohtaan.

Scrum masterin yhdenvertainen asema tiimin kanssa antaa hänelle hyvät edellytykset toimia tehtävässään. Scrum Masterin tehtävä on yhteistyössä tiimin kanssa kehittää tiimiä toimimaan entistäkin paremmin. Päätökset ja muutokset tekee tiimi, mutta Scrum Master johdattelee tiimiä kohti päätöksiä kysymällä ja kyseenalaistamalla.

\subsection{Agile-valmentaja}

Agile-valmentajan ero Scrum Masteriin verrattuna on usein hieman epäselvä, koska Scrum Master saattaa hyvin usein toimia Agile-valmentajana \citep{grosjean2010}. Mielestäni eron voisi kuitenkin tiivistää niin, että Agile-valmentaja on kuin Scrum Master, mutta laajemmalla perspektiivillä.

Scrum Master keskittyy useimmiten tiiminsä tai tiimiensä valmentamiseen, mutta Agile-valmentaja keskittyy yksilön, tiimin tai koko organisaation valmentamiseen \citep{grosjean2010}. 

Agile-valmentajalla on Scrum Masteria laajempi kokemus sekä syvällinen ymmärtäminen ketterien menetelmien ideologiasta ja soveltamisesta käytännössä. Agile-valmentaja ei keskity pelkästään tiettyyn prosessimalliin, kuten esimerkiksi Scrumiin, vaan pyrkii edistämään ketterien menetelmien ideologiaa koko organisaation toiminnassa.

Agile-valmentajan suhde muihin organisaation jäseniin on samankaltainen kuin Scrum Masterilla. Agile-valmentaja ei käytä toiminnassaan käskyttämisvaltaa vaan valmentamalla ja kysymyksiä esittämällä pyrkii edistämään ketterien menetelmien käyttöä.

\subsection{Tuotteen omistaja}

Tuotteen omistaja on asiakkaan edustajana osa projektitiimiä. Tuotteen omistajan tehtäviin kuuluu muun muassa tuotteen kehitysjonon (product backlog) omistaminen, vaatimusten määrittely, liiketoimintaympäristön selittäminen tiimille sekä toimitettujen ominaisuuksien hyväksyminen \citep{augustine2005}.

Asiakkaan, eli maksavan puolen edustajana tuotteen omistajalla on tiettyä päätösvaltaa tiimiä kohtaan. Tuotteen omistajalla on valtaa päättä mitä ja missä järjestyksessä tehdään, mutta ei valtaa päättää miten tehdään.

Tuotteen omistajan tulee olla kuitenkin tiiviissä yhteistyössä tiimin kanssa. Tällä tavoin yhteistyön avulla pyritään pääsemään parhaaseen lopputulokseen, ei niinkään määräämisen avulla.

% Discuss the differences between the following roles in software development projects: 
% - project manager
% - product owner
% - scrum master
% - agile coach

% What are the most important tasks of each of these roles? 

% How are these roles different? 

% Discuss especially from the point of view on how these different roles are involved in leading and managing people.

% Business vastuu?!?!?!

\section{Ihmisten johtaminen ohjelmistoalalla}

Perinteisesti organisaatiorakenteet ovat perustuneet hierarkiaan, jossa johtajilla on alaisia, joilla taas on omia alaisiaan. Johtajilla on käskyvaltaa alaisiinsa nähden ja käskeminen onkin ollut pääasiallinen johtamisen keino. Johtajat ovat myös vastuussa alaistensa tuloksesta, joten tämä vastuu kompensoidaan heille alaisia suuremmalla palkalla.

Perinteiseltä näkökannalta katsottuna ketterien menetelmien roolit saattavat vaikuttaa oudoilta. Miten tiimi voi olla vastuussa projektin onnistumisesta? Kuka vahtii heitä, jos projektimanageri puuttuu? Kuka vahtii Scrum Masterin toimintaa? Entä kuka vahtii, että Agile-valmentaja toimii tehokkaasti? 

Ohjelmistoala on tietotyötä, joka vaatii monia älyllisiä taitoja kuten loogista ongelmanratkaisukykyä, oppimista ja luovuutta. Kuten Daniel Pinkin puheen pohjalta tehdyssä animaatiossa todettiin, tietotyössä perinteinen keppiä ja porkkanaa -menetelmä ei toimi. Tietotyössä suuri palkka ei ole hyvä motivaattori \citep{rsaanimate}.

Mikä sitten motivoi? Samassa animaatiossa todetaan, että yksi vahvimmista motivaattoreista on autonomia \citep{rsaanimate}. Vastuu ja vapaus päättää itse työstään luo vahvan siteen tehtävään työhön. Juuri tästä syystä perinteinen käskyttäminen ei ohjelmistoalalla toimi ja juuri tästä syystä ketterien menetelmien roolit toimivat. Ohjelmistoalan työntekijät ovat omien kokemusteni mukaan keskimääräistä korkeammin koulutettuja, motivoituneita ja hyvin harrastuneita. Tästä syystä tiimille voidaan antaa vastuu projektin onnistumisesta ja samalla vastuun lisääntyessä projektin tulokset kasvavat motivaation kohotessa.

Ohjelmistoalalla lähes poikkeuksetta työ tehdään aina tiimeissä. Tiimien muodostettaessa ja johdettaessa saattaa syntyä useita ongelmia, jotka johtavat tiimityöskentelyn epäonnistumiseen. \citet{mcconnell1996} listaa muun muassa seuraavat: yhteisen näkemyksen puute, tiimi-identiteetin puute, tunnustuksen puute, esteet tuottavuudelle, kommunikaation puute, luottamuksen puute ja ongelmat henkilökemioissa. 

Erityisesti ongelmat henkilökemioissa nousevat usein esille ohjelmistoalalla. Luulen, että yksi syy tähän lienee työntekijöiden harrastuneisuus ja intohimo alaa kohtaan. Usein erilaiset näkemykset ja työskentelytavat saattavat synnyttää pieniä riitoja tiimeissä, ja jos asiaan ei puututa, haittaa tämä tiimin toimintaa pitkällä tähtäimellä. Toisinaan tiimin jäsen saattaa olla hyvin suojeleva omaa työtänsä kohtaan, eikä suostu ottamaan vastaan kritiikkiä. On todettu, että suurin moite, jota tiimin jäsenet sanovat tiimin vetäjistään on, että he eivät ole puuttuneet tiimin ongelmayksilöiden käyttäytymiseen.



% Tiimityö
% Henkilökemiat
% Harrastuneisuus, intohimo
% Luottamuksen puute, vaikea nähdä luomusta konkreettisesti
% Motivointi: Legacy code
% 

% Discuss what can be difficult when managing/leading software developers and building software development teams. 

% Feel free to add questions to our visiting lecturers regarding this topic in the end of your answer.

\citep{grosjean2010}
\citep{appelo2012}
\citep{rsaanimate}
\citep{adkins2010}
\citep{augustine2005}
\citep{mcconnell1996}

\bibliographystyle{plainnat}
\bibliography{ref}

\end{document}